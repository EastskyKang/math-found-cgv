%%%%%%%%%%%%%%%%%%%%%%%%%%%%%%%%%%%%%%%%%
% Short Sectioned Assignment
% LaTeX Template
% Version 1.0 (5/5/12)
%
% This template has been downloaded from:
% http://www.LaTeXTemplates.com
%
% Original author:
% Frits Wenneker (http://www.howtotex.com)
%
% License:
% CC BY-NC-SA 3.0 (http://creativecommons.org/licenses/by-nc-sa/3.0/)
%
%%%%%%%%%%%%%%%%%%%%%%%%%%%%%%%%%%%%%%%%%

%----------------------------------------------------------------------------------------
%	PACKAGES AND OTHER DOCUMENT CONFIGURATIONS
%----------------------------------------------------------------------------------------

\documentclass[paper=a4, fontsize=11pt]{scrartcl} % A4 paper and 11pt font size

\usepackage[T1]{fontenc} % Use 8-bit encoding that has 256 glyphs
\usepackage{fourier} % Use the Adobe Utopia font for the document - comment this line to return to the LaTeX default
\usepackage[english]{babel} % English language/hyphenation
\usepackage{amsmath,amsfonts,amsthm} % Math packages

\usepackage{lipsum} % Used for inserting dummy 'Lorem ipsum' text into the template

\usepackage{sectsty} % Allows customizing section commands
\allsectionsfont{\centering \normalfont\scshape} % Make all sections centered, the default font and small caps

\usepackage{fancyhdr} % Custom headers and footers

% my packages
\usepackage{commath}

\pagestyle{fancyplain} % Makes all pages in the document conform to the custom headers and footers
\fancyhead{} % No page header - if you want one, create it in the same way as the footers below
\fancyfoot[L]{} % Empty left footer
\fancyfoot[C]{} % Empty center footer
\fancyfoot[R]{\thepage} % Page numbering for right footer
\renewcommand{\headrulewidth}{0pt} % Remove header underlines
\renewcommand{\footrulewidth}{0pt} % Remove footer underlines
\setlength{\headheight}{13.6pt} % Customize the height of the header

\numberwithin{equation}{section} % Number equations within sections (i.e. 1.1, 1.2, 2.1, 2.2 instead of 1, 2, 3, 4)
\numberwithin{figure}{section} % Number figures within sections (i.e. 1.1, 1.2, 2.1, 2.2 instead of 1, 2, 3, 4)
\numberwithin{table}{section} % Number tables within sections (i.e. 1.1, 1.2, 2.1, 2.2 instead of 1, 2, 3, 4)

\setlength\parindent{0pt} % Removes all indentation from paragraphs - comment this line for an assignment with lots of text

\newcommand{\filename}[1]{\textbf{\textit{#1}}}
\newcommand{\funcname}[1]{\textbf{#1}}


%----------------------------------------------------------------------------------------
%	TITLE SECTION
%----------------------------------------------------------------------------------------

\newcommand{\horrule}[1]{\rule{\linewidth}{#1}} % Create horizontal rule command with 1 argument of height

\title{	
\normalfont \normalsize 
\textsc{Mathematical foundations of computer graphics and vision} \\ [25pt] % Your university, school and/or department name(s)
\horrule{0.5pt} \\[0.4cm] % Thin top horizontal rule
\huge Exercise 1. Robust Estimation and Optimization \\ % The assignment title
\horrule{2pt} \\[0.5cm] % Thick bottom horizontal rule
}

\author{Dongho Kang} % Your name

\date{\normalsize March 12, 2017} % Today's date or a custom date

\begin{document}

\maketitle % Print the title

%----------------------------------------------------------------------------------------
%	README
%----------------------------------------------------------------------------------------

The "code" directory contains followings:

\begin{itemize}
	\item \filename{part1.m} \quad Script .m file for part1.
	\item \filename{part2.m} \quad Script .m file for part2.
	\item \filename{functions dir} \quad Function .m files 
		\begin{itemize}
			\item ...
			\item ...
		\end{itemize}
\end{itemize}

%----------------------------------------------------------------------------------------
%	PROBLEM 1
%----------------------------------------------------------------------------------------

\section{exercise part 1: ransac for circle fitting}

%----------------------------------------------------------------------------------------
%	DESCRIPTION
%----------------------------------------------------------------------------------------
\subsection{Description}

In exercise part 1, RANSAC algorithm was used for circular model fitting with $N$ number of 2D data points which were corrupted by noise and certain ration of outliers. The implementation of part 1 corresponds to the .m script file \filename{part1.m} and the script consists of several parts which can be executed separately: 

\begin{itemize}
\item Data generation and verification
\item Run RANSAC algorithm for circular model fitting
\item Run exhaustive search for circular model fitting
\item Plot the result
\end{itemize}  

%----------------------------------------------------------------------------------------
\subsubsection{Data generation and verification}

For certain outlier ratio $r  (\%)$ and $N$ number of 2D data points, the number of inliers ($n_{inlier}$)and outliers($n_{outlier}$) are as follows:

\begin{center}
$n_{inlier} = N \times \frac{1 - r}{100} $ \\
$n_{outlier} = N \times \frac{r}{100} $  
\end{center}

\begin{center}
$N = 100$, $r \in \{5, 20, 30, 70\}$
\end{center}

In order to generate $n_{inlier}$ inliers and $n_{outlier}$ outliers, the function \funcname{GenerateInlierData} and \funcname{GenerateOutlierData} are implemented. \funcname{GenerateInlierData} generates noise-added data points which is not over the inlier distance threshold $\tau = 0.1$ from synthetic model and \funcname{GenerateOutlierData} generates data points which is over $\tau$. It can be described as follows. $d\big(p_{1}, p_{2}\big)$ is euclidean distance function between two 2D points $p_{1} and p_{2}$:

\begin{center}
$\abs{d \big( (x_{i}, y_{i}), (x_{c}, y_{c}) \big) - R_{c}} \leq \tau$  \qquad for $(x_{i}, y_{i}) \in Set_{inlier}$ \\
$\abs{d \big( (x_{j}, y_{j}), (x_{c}, y_{c}) \big) - R_{c}} > \tau $ \qquad for $(x_{i}, y_{i}) \in Set_{outlier}$ 
\end{center}


After generating inliers/outliers, verified whether generated inliers is ~ threshold and also generated outliers . For this purpose, the function \funcname{VerifyInlier} is used. The specific descriptions for each functions are following:

\paragraph{Inlier generation}

A 2D point $(\tilde{x_{i}}, \tilde{y_{i}})$ on the circular model $(x-x_{c})^{2} + (y-y_c)^{2} = R_{c}^{2}$ which of center $(x_{c}, y_{c})$, and radius $R_{c}$ can be described as follows:

\begin{center}
$(\tilde{x_{i}}, \tilde{y_{i}}) = (x_{c}, y_{c}) + R_{c} \times (\cos \theta_{i}, \sin \theta_{i})$ \qquad $\theta_{i} \in [0, 2 \pi]$, $i \in \{1, 2, \dots n_{inlier} \}$
\end{center}

Thus, picked $n_{inlier}$ number of random float numbers from the range of  $[0, 2 \pi]$ by \funcname{rand} and as using these numbers as $\theta_{i}$, generated $(x_{i}, y_{i})$ for $i \in \{1, 2, \dots n_{inlier} \}$.  

Random noise $(\sigma_{x_{i}}, \sigma_{y_{i}})$ is added on the data point $(x_{i}, y_{i})$ as follows:

\begin{center}
$(x_{i}, y_{i}) = (\tilde{x_{i}}, \tilde{y_{i}}) + (\sigma_{x_{i}}, \sigma_{y_{i}})$ \\
s.t. $\sigma_{x_{i}}, \sigma_{y_{i}} \in [-0.1, 0.1], \quad \abs{d((\tilde{x}_{i}, \tilde{y}_{i}), (x_{c}, y_{c})) - R_{c}} \leq \tau $
\end{center}

To implement this, two random float numbers by \funcname{rand} were picked for $(\sigma_{x_{i}}, \sigma_{y_{i}})$ iteratively, until it satisfies $\abs{d((x_{i}, y_{i}), (x_{c}, y_{c})) - R_{c}} \leq \tau$ and repeated it for $i=1...n_{inlier}$ to generate $n_{inlier}$ number of inliers.

\paragraph{outlier generation}

In order to generate one outlier point $(x_{j}, y_{j})$, two random float numbers were generated by \funcname{rand} in the domain of $[-10, 10] \times [-10, 10]$ and generated $(x_{j}, y_{j})$ was checked whether it satisfies $\abs{d((x_{i}, y_{i}), (x_{c}, y_{c})) - R_{c}} > \tau$. If not, generated $(x_{j}, y_{j})$ again. Repeated this for $n_{outlier}$ times to generate $n_{outlier}$ number of outliers.

\paragraph{data verification}

Checked whether synthesized inliers are indeed inliers and outliers are indeed outliers. The function \funcname{VerifyInlier} was used. The function calculates $\abs{d((\tilde{x}_{i}, \tilde{y}_{i}), (x_{c}, y_{c})) - R_{c}}$ for each data points and check if there is a inlier point which of $\abs{d((\tilde{x}_{i}, \tilde{y}_{i}), (x_{c}, y_{c})) - R_{c}} > \tau$ or an outlier point which of $\abs{d((\tilde{x}_{i}, \tilde{y}_{i}), (x_{c}, y_{c})) - R_{c}} \leq \tau$. In this case, error is returned and the script halts. 

%----------------------------------------------------------------------------------------
\subsubsection{RANSAC for circular model fitting}



\subsection{Running}

parameters
debug


\subsection{Result}

\begin{align} 
\begin{split}
(x+y)^3 	&= (x+y)^2(x+y)\\
&=(x^2+2xy+y^2)(x+y)\\
&=(x^3+2x^2y+xy^2) + (x^2y+2xy^2+y^3)\\
&=x^3+3x^2y+3xy^2+y^3
\end{split}					
\end{align}

Phasellus viverra nulla ut metus varius laoreet. Quisque rutrum. Aenean imperdiet. Etiam ultricies nisi vel augue. Curabitur ullamcorper ultricies

%------------------------------------------------

\subsection{Heading on level 2 (subsection)}

Lorem ipsum dolor sit amet, consectetuer adipiscing elit. 
\begin{align}
A = 
\begin{bmatrix}
A_{11} & A_{21} \\
A_{21} & A_{22}
\end{bmatrix}
\end{align}
Aenean commodo ligula eget dolor. Aenean massa. Cum sociis natoque penatibus et magnis dis parturient montes, nascetur ridiculus mus. Donec quam felis, ultricies nec, pellentesque eu, pretium quis, sem.

%------------------------------------------------

\subsubsection{Heading on level 3 (subsubsection)}

\lipsum[3] % Dummy text

\paragraph{Heading on level 4 (paragraph)}

\lipsum[6] % Dummy text

%----------------------------------------------------------------------------------------
%	PROBLEM 2
%----------------------------------------------------------------------------------------

\section{Lists}

%------------------------------------------------

\subsection{Example of list (3*itemize)}
\begin{itemize}
	\item First item in a list 
		\begin{itemize}
		\item First item in a list 
			\begin{itemize}
			\item First item in a list 
			\item Second item in a list 
			\end{itemize}
		\item Second item in a list 
		\end{itemize}
	\item Second item in a list 
\end{itemize}

%------------------------------------------------

\subsection{Example of list (enumerate)}
\begin{enumerate}
\item First item in a list 
\item Second item in a list 
\item Third item in a list
\end{enumerate}

%----------------------------------------------------------------------------------------

\end{document}